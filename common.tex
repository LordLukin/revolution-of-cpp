\usepackage[utf8]{inputenc}
\usepackage{listings,bera}
\usepackage{multicol}
\usepackage{graphicx}
\usepackage{caption}
%\usepackage{soul}
%\usepackage{tikz}
%\usepackage{mathtools}
%\usepackage{xcolor}
\usepackage[absolute,overlay]{textpos}

% comment this out to disable outline slides, between sections
%\AtBeginSection[]
%{
%\begin{frame}<beamer>
%\frametitle{Outline for section \thesection}
%\tableofcontents[currentsection]
%% \begin{center}
%% \textbf{
%% \fontsize{40}{50}\selectfont
%% \insertsection
%% }
%% \end{center}
%\end{frame}
%}

%\newcommand{\listingNormal}
%{
\lstset
{
  language=C++,
  numbers=left,
  basicstyle=\ttfamily\scriptsize,
  keywordstyle=\color{blue}\ttfamily,
  stringstyle=\color{red}\ttfamily,
  commentstyle=\color{cyan}\ttfamily,
  morecomment=[l][\color{magenta}]{\#},
  showlines=true % do NOT ignore any whitespaces in the file!
}
%}

\newcommand{\listingNoLineNum}
{
\lstset{numbers=none}
}

% \slide{name}{content}
\newcommand{\slide}[2]
{
\subsection*{#1}
\begin{frame}
\frametitle{#1}
%\listingNormal
#2
\end{frame}
}


\newcommand{\sectionSlide}[3]
{
\section{#1}
{
\usebackgroundtemplate{
  \parbox[b][\paperheight][b]{\paperwidth}{\centering\includegraphics[width=#3]{#2}}
}
\begin{frame}
\frametitle{#1}
\end{frame}
}
}

\newcommand{\imageSlide}[1]
{
{
\usebackgroundtemplate
{
  \parbox[b][\paperheight][b]{\paperwidth}
  {
    \centering\includegraphics[height=\paperheight]{#1}
  }
}
\begin{frame}[plain]
\end{frame}
}

}

% see http://latexcolor.com for others
\definecolor{celadon}{rgb}{0.67, 0.88, 0.69}
\lstdefinestyle{listingHighlight}{ backgroundcolor=\color{celadon} }
\newcounter{listingHighlightLineCounter}

% \emphLine{fromLine}{toLine}{fileName}
\newcommand{\highlightedListing}[3]
{
  \setcounter{listingHighlightLineCounter}{#1}
  \addtocounter{listingHighlightLineCounter}{-1}

  % when first line is to be highlighted, first section cannot be used
  \ifnumequal{#1}{1}
  {
    \lstinputlisting[belowskip=0pt,
                     linerange={#1-#2},
                     style=listingHighlight,
                     firstnumber=1]
                    {#3}
  }
  {
    \lstinputlisting[belowskip=0pt,
                     linerange={1-\value{listingHighlightLineCounter}}]
                    {#3}
    \lstinputlisting[belowskip=0pt,
                     aboveskip=0pt,
                     linerange={#1-#2},
                     style=listingHighlight,
                     firstnumber=last]
                    {#3}
  }
  \setcounter{listingHighlightLineCounter}{#2}
  \addtocounter{listingHighlightLineCounter}{1}

  \lstinputlisting[aboveskip=0pt,
                   firstline=\the\value{listingHighlightLineCounter},
                   firstnumber=last]
                  {#3}
}

