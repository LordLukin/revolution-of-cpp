\usepackage[utf8]{inputenc}
\usepackage{listings,bera}
\usepackage{multicol}
\usepackage{graphicx}
\usepackage{caption}
\usepackage{soul}
\usepackage{tikz}
\usepackage{mathtools}
\usepackage{xcolor}
\usepackage[absolute,overlay]{textpos}

\usepackage[TS1,T1]{fontenc}
\usepackage{array, booktabs}
\usepackage{graphicx}
\usepackage{colortbl}
\DeclareCaptionFont{blue}{\color{blue}}

% comment this out to disable outline slides, between sections
%\AtBeginSection[]
%{
%\begin{frame}<beamer>
%\frametitle{Outline for section \thesection}
%\tableofcontents[currentsection]
%% \begin{center}
%% \textbf{
%% \fontsize{40}{50}\selectfont
%% \insertsection
%% }
%% \end{center}
%\end{frame}
%}

%\newcommand{\listingNormal}
%{
\lstset
{
  language=C++,
  numbers=left,
  basicstyle=\ttfamily\scriptsize,
  keywordstyle=\color{blue}\ttfamily,
  stringstyle=\color{red}\ttfamily,
  commentstyle=\color{cyan}\ttfamily,
  morecomment=[l][\color{magenta}]{\#},
  showlines=true % do NOT ignore any whitespaces in the file!
}
%}

\newcommand{\listingNoLineNum}
{
\lstset{numbers=none}
}

% \slide{name}{content}
\newcommand{\slide}[2]
{
\subsection*{#1}
\begin{frame}
\frametitle{#1}
%\listingNormal
#2
\end{frame}
}


\newcommand{\foo}{\color{blue}\makebox[0pt]{{\large \textbullet}}\hskip-0.5pt\vrule width 1pt\hspace{\labelsep}}

% \slide{name}{content}
\newcommand{\timelineSlide}[3]
{
    \subsection*{#1}
    \begin{frame}
        \frametitle{#1}
        %\listingNormal
        \begin{columns}
            \begin{column}{0.73\textwidth}
                #2
            \end{column}
            \begin{column}{0.27\textwidth}
                \begin{table}
                    %    \renewcommand\arraystretch{1.4}\arrayrulecolor{blue}
                    %    \captionsetup{singlelinecheck=false, font=blue, labelfont=sc, labelsep=quad}
                    %    \caption{Timeline}\vskip -1.5ex
                    \begin{tabular}{@{\,}r <{\hskip 2pt} !{\foo} >{\raggedright\arraybackslash}p{2.6cm}}
                        %        \toprule
                        %        \addlinespace[1.5ex]
                        #3
                    \end{tabular}
                \end{table}
                \vspace{\fill}
            \end{column}
        \end{columns}
    \end{frame}
}

% \slide{name}{content}
\newcommand{\timeSlide}[3]
{
    \subsection*{#1}
    \begin{frame}
        \frametitle{#1}
        %\listingNormal
        \begin{columns}
            \begin{column}{0.73\textwidth}
                #2
            \end{column}
            \begin{column}{0.27\textwidth}
                \begin{tikzpicture}
                    \draw[thick,->,blue] (0,0.2)--(0,-6.8) % x axis
                    node(1979)[pos=1/42,fill,circle,inner sep=2pt]{}
                    node(1981)[pos=3.5/42,fill,circle,inner sep=2pt]{}
                    node(1983)[pos=6/42,fill,circle,inner sep=2pt]{}
                    node(1984)[pos=7.25/42,fill,circle,inner sep=2pt]{}
                    node(1985)[pos=8.5/42,fill,circle,inner sep=2pt]{}
                    %node(1986)[pos=8/42,fill,circle,inner sep=2pt]{}
                    %node(1987)[pos=9/42,fill,circle,inner sep=2pt]{}
                    node(1989)[pos=11/42,fill,circle,inner sep=2pt]{}
                    %node(1990)[pos=12/42,fill,circle,inner sep=2pt]{}
                    node(1991)[pos=13/42,fill,circle,inner sep=2pt]{}
                    node(1993)[pos=15/42,fill,circle,inner sep=2pt]{}
                    node(1998)[pos=20/42,fill,circle,inner sep=2pt]{}
                    node(2003)[pos=25/42,fill,circle,inner sep=2pt]{}
                    node(2011)[pos=31/42,fill,circle,inner sep=2pt]{}
                    node(2014)[pos=34/42,fill,circle,inner sep=2pt]{}
                    node(2017)[pos=37/42,fill,circle,inner sep=2pt]{}
                    node(2020)[pos=40/42,fill,circle,inner sep=2pt]{};
                    #3
                \end{tikzpicture}
            \end{column}
        \end{columns}
    \end{frame}
}


\newcommand{\sectionSlide}[3]
{
\section{#1}
{
\usebackgroundtemplate{
  \parbox[b][\paperheight][b]{\paperwidth}{\centering\includegraphics[width=#3]{#2}}
}
\begin{frame}
\frametitle{#1}
\end{frame}
}
}

\newcommand{\imageSlide}[1]
{
{
\usebackgroundtemplate
{
  \parbox[b][\paperheight][b]{\paperwidth}
  {
    \centering\includegraphics[height=\paperheight]{#1}
  }
}
\begin{frame}[plain]
\end{frame}
}

}

% see http://latexcolor.com for others
\definecolor{celadon}{rgb}{0.67, 0.88, 0.69}
\lstdefinestyle{listingHighlight}{ backgroundcolor=\color{celadon} }
\newcounter{listingHighlightLineCounter}

% \emphLine{fromLine}{toLine}{fileName}
\newcommand{\highlightedListing}[3]
{
  \setcounter{listingHighlightLineCounter}{#1}
  \addtocounter{listingHighlightLineCounter}{-1}

  % when first line is to be highlighted, first section cannot be used
  \ifnumequal{#1}{1}
  {
    \lstinputlisting[belowskip=0pt,
                     linerange={#1-#2},
                     style=listingHighlight,
                     firstnumber=1]
                    {#3}
  }
  {
    \lstinputlisting[belowskip=0pt,
                     linerange={1-\value{listingHighlightLineCounter}}]
                    {#3}
    \lstinputlisting[belowskip=0pt,
                     aboveskip=0pt,
                     linerange={#1-#2},
                     style=listingHighlight,
                     firstnumber=last]
                    {#3}
  }
  \setcounter{listingHighlightLineCounter}{#2}
  \addtocounter{listingHighlightLineCounter}{1}

  \lstinputlisting[aboveskip=0pt,
                   firstline=\the\value{listingHighlightLineCounter},
                   firstnumber=last]
                  {#3}
}

